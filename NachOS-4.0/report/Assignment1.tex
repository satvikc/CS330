\documentclass{article}
\usepackage{amssymb,amsmath}
\usepackage{ifxetex,ifluatex}
\ifxetex
  \usepackage{fontspec,xltxtra,xunicode}
  \defaultfontfeatures{Mapping=tex-text,Scale=MatchLowercase}
\else
  \ifluatex
    \usepackage{fontspec}
    \defaultfontfeatures{Mapping=tex-text,Scale=MatchLowercase}
  \else
    \usepackage[utf8]{inputenc}
  \fi
\fi
\ifxetex
  \usepackage[setpagesize=false, % page size defined by xetex
              unicode=false, % unicode breaks when used with xetex
              xetex]{hyperref}
\else
  \usepackage[unicode=true]{hyperref}
\fi
\hypersetup{breaklinks=true, pdfborder={0 0 0}}
\setlength{\parindent}{0pt}
\setlength{\parskip}{6pt plus 2pt minus 1pt}
\setlength{\emergencystretch}{3em}  % prevent overfull lines
\setcounter{secnumdepth}{0}


\begin{document}

\section{Assignment 1}

\subsection{Satvik Chauhan (Y9521) , Pankaj More (Y9402)}

\begin{itemize}
\item
  Changes in Nachos

  \begin{itemize}
  \item
    Changes in Address Space

    \begin{itemize}
    \item
      Nachos was a single process system , so it used the whole of its
      memory for the address space of a single process .
    \item
      Every address space consist of pagetable which is made up of pages
      .
    \item
      The first page of the address space starts with virtual address 0
      ,but it has some mapping to the physical address on the main
      memory.
    \item
      In the original nachos both virtual address and the physical
      address were same ( equals 0) as it had to deal with only one
      process .
    \item
      The class Bitmap from Bitmap.h was used to create a bitmap for the
      pages in the physical memory and to find the empty physical
      address while creating a new address space.
    \item
      Initially after creating the new address space , the whole
      physical memory was wiped out to 0 . So we had to change it so
      that it only assigns 0 to its own address space bits.
    \item
      We had to translate the virtual address to the corresponding
      physical address while copying any file to the address space.
    \item
      To handle the specific need of the Fork system call in which we
      had to copy the current address space to a newly created address
      space we used \emph{constructor overloading} to create a new
      address space constructor deep copies the address space from the
      address space pointer passed to it.
    \item
      We had to move the pagetable creation step from the original
      constructor to the Load function . So it creates new address space
      when Load function is called and assigns required amount of pages
      for it .
    \item
      The Address space destructor has to take care to free up the
      memory and mark the corresponding pages in the pagetable bitmap as
      dirty so that the space can be reused.
    \item
      We also created a class element called id which assigns a unique
      id to each process based on the starting page of its address
      space.
    \end{itemize}
  \item
    System calls

    \begin{itemize}
    \item
      \emph{Fork}

      \begin{itemize}
      \item
        Fork creates a new kernel level thread
      \item
        Copies the current address space to the address space of the new
        thread .
      \item
        Also keep the registers to be used by the child process .
      \item
        Puts child on the ready queue and starts the child process.
      \item
        It returns 0 in the child process and the \emph{PID} of the
        child in the parent process.
      \end{itemize}
    \item
      There are two implementations of the exec system call described
      below

      \begin{itemize}
      \item
        \emph{Exec2}

        \begin{itemize}
        \item
          To be used with Fork system call only .
        \item
          It replaces the current address space with the program to be
          loaded .
        \item
          Resets all the registers and thus moves the program counter to
          the beginning .
        \item
          The control returns to the parent process after the child
          process finishes.
        \end{itemize}
      \item
        \emph{Exec}

        \begin{itemize}
        \item
          An amalgamation of the above Exec2 and Fork calls .
        \item
          Instead of calling Fork and then Exec2 , We can directly call
          Exec with the executable of the new process and it directly
          loads and starts the new process.
        \item
          The control returns to the parent process after the child
          process finishes .
        \end{itemize}
      \end{itemize}
    \item
      \emph{Exit}

      \begin{itemize}
      \item
        Exit system call is called automatically whenever a program
        exits .
      \item
        Handles the cleanup of the system .
      \item
        It stops the thread if it is not the main thread .
      \item
        Since the last thread can not stop itself , so we can not just
        finish the last \emph{main} thread .
      \item
        So if the pid of the thread calling main is 0 (0 means it is the
        main thread) it calls nachos Halt instead of finishing the
        thread.
      \end{itemize}
    \item
      SysStats
      \begin{itemize}
      \item
        SysStats internally calls SysStatsCall() function defined in
        exception.cc .
      \item
        The structures ProcInfo and SysInfo as given in the question
        have been defined in machine/stats.h with a few modifications
        (for optimisation and convienince) which are only internal to
        the code. The external api for calling it is consistent with the
        question.
      \item
        A global mysysinfo structure is created as a member of the
        kernel object and it can be accessed as
        kernel-\textgreater{}mysysinfo .
      \item
        Since the tick variables are already defined and are working in
        stats.h , our corresponding ticks in SysInfo structure are just
        pointers to those pre-defined ticks. When the kernel is
        initialised , and kernel-\textgreater{}stats is created , we
        setup the corresponding pointer references in
        Kernel::Initialize() .
      \item
        We create the array of ProcInfos during Thread creation. When
        Thread::Thread() constructor is called, we simply create an new
        ProcInfo structure and add it to the proc array in mysysinfo and
        setup the corresponding pointer references for thread name and
        thread status.
      \item
        kernel-\textgreater{}mysysinfo-\textgreater{}proc refers to an
        array of ProcInfo structures which store the status of each
        procces name and status.
      \item
        Our Nachos user processes are nothing but Thread objects defined
        in /threads/threads.h .
      \item
        When a process is just created , its status is JUST\_CREATED.
      \item
        When the Nachos System is running the process , it sets its
        status to RUNNING.
      \item
        When the process is Yield , its status becomes READY.
      \item
        When the process is finished/killed , it becomes BLOCKED.
      \item
        When SysStats system call is made , it prints all the processes
        (both running and killed) along with their status. Moreover , it
        also prints out the relevant ticks details on the console.
      \end{itemize}
    \end{itemize}
  \item
    Assignment1.c
    \begin{itemize}
    \item
      Assignment1.c uses Exec2 system call which comfirms to the spec
      given in the question.
    \item
      It basically calls fork and execute on each programs in the progs
      array.
    \end{itemize}
  \item
    otherExec.c
    \begin{itemize}
    \item
      It uses the Exec system call. It is different from the one asked
      in question.
    \item
      It does a thread fork and then loads the program simultaneously on
      the same Exec system call.
    \end{itemize}
  \end{itemize}
\end{itemize}

\end{document}
